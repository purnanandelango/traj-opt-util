\documentclass[letterpaper,11pt]{article}
\usepackage[margin=0.75in]{geometry}
\usepackage[utf8]{inputenc}
\usepackage{mathpazo}
\usepackage[T1]{fontenc}
\usepackage{amsmath,amssymb,bm}
\setlength{\parskip}{0pt}
\setlength{\parindent}{0pt}

\title{PIPG Module}
\author{Purnanand Elango}

\begin{document}

\maketitle 

\section{Template Optimal Control Problem}

\begin{subequations}
\begin{align}
    \operatorname{minimize}~~&~\sum_{t=1}^Nx_t^\top Q_t x_t +q_t^\top x_t + u_t^\top R_t u_t + r_t^\top u_t,\\
    \operatorname{subject~to}~~&~x_{t+1} = A_t x_t + B_t^{-}u_t + B_{t+1}^+u_{t+1} + c_t, & & t = 1,\ldots,N-1,\\
    &~x_t\in\mathbb{D}^x_t,~~u_t \in \mathbb{D}^u_t, & & t = 1,\ldots, N.\label{eq:x-u-proj-cnstr}  
\end{align}\label{prb:template-ocp}%
\end{subequations}

To track known state reference $x_t^{\text{ref}}$ and/or a control reference $u^{\text{ref}}_t$, choose $q_t = -2x_t^{\text{ref}}$ and $r_t = -2u^{\text{ref}}_t$. The boundary conditions on states and control are accounted in \eqref{eq:x-u-proj-cnstr}.

\section{Conic Optimization Problem}

\begin{subequations}
\begin{align}
    \operatorname{minimize}~~&~\frac{1}{2}z^\top P  z + p^\top z\\
    \operatorname{subject~to}~~&~Hz-g \in \mathbb{K},\\
    &~z\in\mathbb{D}.    
\end{align}    
\end{subequations}

\section{Extrapolated PIPG (\textsc{xpipg})}

\section{Template Extension to General SOCPs}

% \begin{subequations}
\begin{align*}
    \operatorname{minimize}~~&~\sum_{t=1}^Nx_t^\top Q_t x_t +q_t^\top x_t + u_t^\top R_t u_t + r_t^\top u_t,\\
    \operatorname{subject~to}~~&~~~x_{t+1} = A_t x_t + B_t^{-}u_t + B_{t+1}^+u_{t+1} + c_t, & & t = 1,\ldots,N-1,\\
        & \left. \begin{array}{l} 
        x_t\in\mathbb{D}^x_t,~~u_t \in \mathbb{D}^u_t,\\[0.1cm]
        F^0_t x_t + G^0_t u_t + h^0_t = 0, \\[0.1cm]
        F^1_t x_t + G^1_t u_t + h^1_t \le 0, \\[0.1cm]
        F^2_t x_t + G^2_t u_t + h^2_t \preceq_{2} 0 , \end{array} \right\} & & t = 1,\ldots, N,  
\end{align*}
% \end{subequations}

where $\preceq_2$ is the generalized inequality representing a second-order cone.

\end{document}